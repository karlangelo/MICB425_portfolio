\documentclass[]{article}
\usepackage{lmodern}
\usepackage{amssymb,amsmath}
\usepackage{ifxetex,ifluatex}
\usepackage{fixltx2e} % provides \textsubscript
\ifnum 0\ifxetex 1\fi\ifluatex 1\fi=0 % if pdftex
  \usepackage[T1]{fontenc}
  \usepackage[utf8]{inputenc}
\else % if luatex or xelatex
  \ifxetex
    \usepackage{mathspec}
  \else
    \usepackage{fontspec}
  \fi
  \defaultfontfeatures{Ligatures=TeX,Scale=MatchLowercase}
\fi
% use upquote if available, for straight quotes in verbatim environments
\IfFileExists{upquote.sty}{\usepackage{upquote}}{}
% use microtype if available
\IfFileExists{microtype.sty}{%
\usepackage{microtype}
\UseMicrotypeSet[protrusion]{basicmath} % disable protrusion for tt fonts
}{}
\usepackage[margin=1in]{geometry}
\usepackage{hyperref}
\hypersetup{unicode=true,
            pdfborder={0 0 0},
            breaklinks=true}
\urlstyle{same}  % don't use monospace font for urls
\usepackage{graphicx,grffile}
\makeatletter
\def\maxwidth{\ifdim\Gin@nat@width>\linewidth\linewidth\else\Gin@nat@width\fi}
\def\maxheight{\ifdim\Gin@nat@height>\textheight\textheight\else\Gin@nat@height\fi}
\makeatother
% Scale images if necessary, so that they will not overflow the page
% margins by default, and it is still possible to overwrite the defaults
% using explicit options in \includegraphics[width, height, ...]{}
\setkeys{Gin}{width=\maxwidth,height=\maxheight,keepaspectratio}
\IfFileExists{parskip.sty}{%
\usepackage{parskip}
}{% else
\setlength{\parindent}{0pt}
\setlength{\parskip}{6pt plus 2pt minus 1pt}
}
\setlength{\emergencystretch}{3em}  % prevent overfull lines
\providecommand{\tightlist}{%
  \setlength{\itemsep}{0pt}\setlength{\parskip}{0pt}}
\setcounter{secnumdepth}{0}
% Redefines (sub)paragraphs to behave more like sections
\ifx\paragraph\undefined\else
\let\oldparagraph\paragraph
\renewcommand{\paragraph}[1]{\oldparagraph{#1}\mbox{}}
\fi
\ifx\subparagraph\undefined\else
\let\oldsubparagraph\subparagraph
\renewcommand{\subparagraph}[1]{\oldsubparagraph{#1}\mbox{}}
\fi

%%% Use protect on footnotes to avoid problems with footnotes in titles
\let\rmarkdownfootnote\footnote%
\def\footnote{\protect\rmarkdownfootnote}

%%% Change title format to be more compact
\usepackage{titling}

% Create subtitle command for use in maketitle
\newcommand{\subtitle}[1]{
  \posttitle{
    \begin{center}\large#1\end{center}
    }
}

\setlength{\droptitle}{-2em}
  \title{}
  \pretitle{\vspace{\droptitle}}
  \posttitle{}
  \author{}
  \preauthor{}\postauthor{}
  \date{}
  \predate{}\postdate{}


\begin{document}

\paragraph{Learning objectives:}\label{learning-objectives}

Specific emphasis should be placed on the process used to find the
answer. Be as comprehensive as possible e.g.~provide URLs for web
sources, literature citations, etc.\\
\emph{(Reminders for how to format links, etc in RMarkdown are in the
RMarkdown Cheat Sheets)}

\paragraph{Specific Questions:}\label{specific-questions}

\begin{itemize}
\item
  How many prokaryotic divisions have been described and how many have
  no cultured representatives (microbial dark matter)?

  \begin{itemize}
  \item
    At least 89 bacterial and 20 archaeal phyla are recognized via small
    subunit ribosomal RNA databases, although the true phyla count is
    certainly higher and could range up to 1,500

    \begin{itemize}
    \tightlist
    \item
      As there are prokaryotes that live in the ``shadow biosphere''
      --\textgreater{} which is a hypothetical microbial biosphere
      unknown to life
    \end{itemize}
  \item
    26 of the approximately 52 identifiable major phyla, within the
    domain Bacteria have cultivated representatives

    \begin{itemize}
    \tightlist
    \item
      Thus, 52-26 of the major phyla of Bacteria are uncultured
    \end{itemize}
  \item
    Point is most of the life is uncultured. Only information we have
    about life is from seqeuncing.
  \item
    Specific references for the question above:

    \begin{itemize}
    \item
      Solden, L, Lloyd, K, Wrighton, K. 2016. The bright side of
      microbial dark matter: lessons learned from the uncultivated
      majority. Curr. Opin. Microbiol. 31:217-226. doi:
      10.1016/j.mib.2016.04.020.
    \item
      Youssef, NH, Couger, MB, McCully, AL, Criado, AEG, Elshahed, MS.
      2015. Assessing the global phylum level diversity within the
      bacterial domain: A review. Journal of Advanced Research.
      6:269-282. doi: 10.1016/j.jare.2014.10.005.
    \item
      Rappé, MS, Giovannoni, SJ. 2003. THE UNCULTURED MICROBIAL
      MAJORITY. Annual Reviews in Microbiology. 57:369-394. doi:
      10.1146/annurev.micro.57.030502.090759.
    \end{itemize}
  \end{itemize}
\item
  How many metagenome sequencing projects are currently available in the
  public domain and what types of environments are they sourced from?
\item
  Shot-gun metagenomics:

  \begin{itemize}
  \tightlist
  \item
    Assembly: EULER, IMG -/M
  \item
    Binning: S-GCOM, IMG-RAST
  \item
    Annotation: KEGG, NCBI
  \item
    Analysis: Megan 5
  \end{itemize}
\item
  Marker Gene Metagenomics:

  \begin{itemize}
  \tightlist
  \item
    Standalone software:OTU base
  \item
    Analysis: SILVA
  \item
    Denoising: Amplicon Noise
  \item
    Datapases: Ribosomal Database Project (RDP) rences: from a paper,
    review artile
  \end{itemize}
\item
  What types of on-line resources are available for warehousing and/or
  analyzing environmental sequence information (provide names, URLS and
  applications)?

  \begin{itemize}
  \tightlist
  \item
    IMG/-M (\url{https://img.jgi.doe.gov/m/})
  \item
    MEGAN (\url{https://ab.inf.uni-tuebingen.de/software/megan/})
  \item
    MetaPathways
    (\url{https://hallam.microbiology.ubc.ca/MetaPathways/})
  \end{itemize}
\item
  What is the difference between phylogenetic and functional gene
  anchors and how can they be used in metagenome analysis?

  \begin{itemize}
  \tightlist
  \item
    Phylogenetic:

    \begin{itemize}
    \tightlist
    \item
      veritcal gene transfer
    \item
      carry phylogenetic info
    \item
      taxonomic
    \item
      ideally single copy
    \end{itemize}
  \item
    Functional:

    \begin{itemize}
    \tightlist
    \item
      more horizontal gene transfer
    \item
      identify specific biogeochemical functions associated with
      measureable effects
    \item
      not as useful for phylogentic construction
    \end{itemize}
  \end{itemize}
\item
  What is metagenomic sequence binning? What types of algorithmic
  approaches are used to produce sequence bins? What are some risks and
  opportunities associated with using sequence bins for metabolic
  reconstruction of uncultivated microorganisms?

  \begin{itemize}
  \item
    Binning: process of grouping sequences that comes from a single
    genome
  \item
    Types of algorithms:
  \end{itemize}

  \begin{enumerate}
  \def\labelenumi{\arabic{enumi}.}
  \tightlist
  \item
    Align sequences to database
  \item
    Group to each based on DNA characteristics: GC content, codon usage
  \end{enumerate}

  \begin{itemize}
  \tightlist
  \item
    Risks \& Oppurtunities: Risks:
  \item
    incomplete coverage of genome sequences
  \item
    contamination of different phylogeny. Questions to ask what is
    considered a contamination?

    \begin{itemize}
    \tightlist
    \item
      Threshold should be be minimum 10\% for contamition
    \end{itemize}
  \item
    What is a genome from a metagenome?
  \end{itemize}
\item
  Is there an alternative to metagenomic shotgun sequencing that can be
  used to access the metabolic potential of uncultivated microorganisms?
  What are some risks and opportunities associated with this
  alternative?

  \begin{itemize}
  \tightlist
  \item
    functional screens (biochemical, etc)
  \item
    3rd gene seqeuecing (nanopore)
  \item
    Single sequencing
  \item
    FISH probe
  \end{itemize}
\end{itemize}


\end{document}
